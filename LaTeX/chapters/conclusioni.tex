\chapter{Conclusioni}


In questa tesina è stato analizzato il \textit{Problema del Commesso Viaggiatore} ( \textit{Travelling Salesman Problem}, TSP), un problema di ottimizzazione che richiede di determinare il percorso di costo minimo per visitare un insieme di nodi esattamente una volta, tornando infine al nodo di partenza. 

Sono state analizzate due formulazioni del problema che portano a un numero esponenziale di vincoli per garantire la connettività della soluzione. In seguito è stato implementato l’algoritmo di branching in due versioni che sono state confrontate. 

Infine l’algoritmo, nella sua versione binaria, è stato applicato, facendo vedere i diversi passaggi, al caso di studio di Elon Musk per determinare il percorso di costo minimo per la sua avventura spaziale nel sistema solare.

Forniamo di seguito il link alla nostra  \href{https://github.com/tkachenko0/Traveling_Salesperson_Problem}{repository GitHub} nella quale può essere trovata la nostra implementazione insieme anche alla matrice che abbiamo usato per simulare l'esecuzione.
